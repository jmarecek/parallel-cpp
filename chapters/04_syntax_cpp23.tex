\chapter{The Syntax in C++23}

\section{Threads, Tasks, Coroutines}

\subsection{C++11 Threads}

\raggedbottom
\begin{codebox}[]{\href{https://godbolt.org/z/}{\ExternalLink}}
\footnotesize An example of the use of a C++11 thread.
\tcblower
\cppfile{code_examples/cpp23/thread1.cpp}
\end{codebox}

\raggedbottom
\begin{codebox}[]{\href{https://godbolt.org/z/}{\ExternalLink}}
\footnotesize An example of the use of a C++11 thread.
\tcblower
\cppfile{code_examples/cpp23/thread2.cpp}
\end{codebox}

\subsection{C++20 Threads}

\raggedbottom
\begin{codebox}[]{\href{https://godbolt.org/z/}{\ExternalLink}}
\footnotesize An example of the use of jthread.
\tcblower
\cppfile{code_examples/cpp23/jthread1.cpp}
\end{codebox}

\raggedbottom
\begin{codebox}[]{\href{https://godbolt.org/z/}{\ExternalLink}}
\footnotesize An example of the use of jthread.
\tcblower
\cppfile{code_examples/cpp23/jthread2.cpp}
\end{codebox}

\raggedbottom
\begin{codebox}[]{\href{https://godbolt.org/z/}{\ExternalLink}}
\footnotesize An example of the use of jthread.
\tcblower
\cppfile{code_examples/cpp23/jthread3.cpp}
\end{codebox}

\raggedbottom
\begin{codebox}[]{\href{https://godbolt.org/z/}{\ExternalLink}}
\footnotesize An example of the use of jthread.
\tcblower
\cppfile{code_examples/cpp23/jthread4.cpp}
\end{codebox}

\raggedbottom
\begin{codebox}[]{\href{https://godbolt.org/z/}{\ExternalLink}}
\footnotesize An example of the use of jthread.
\tcblower
\cppfile{code_examples/cpp23/jthread5.cpp}
\end{codebox}

\raggedbottom
\begin{codebox}[]{\href{https://godbolt.org/z/}{\ExternalLink}}
\footnotesize An example of the use of jthread.
\tcblower
\cppfile{code_examples/cpp23/jthread6.cpp}
\end{codebox}

\subsection{Tasks}

\subsection{Coroutines}

Although the coroutine has no return statement, it has a return type, which needs to be defined by the programmer in C++20. In C++23 and subsequent versions, we hope to see some standard types such as:

\raggedbottom
\begin{codebox}[]{}
\footnotesize An example of the use of coroutines, which currently does not compile in GCC 12.2.
\tcblower
\cppfile{code_examples/cpp23/coroutines1.cpp}
\end{codebox}

Independently of the return type, we have the state stored in heap, using template class ``std::coroutine_handle'' as the equivalent of a pointer and its method ``destroy'' as equivalent to a free. 

We also see 
\begin{itemize}
\item ``co_await std::suspend_always{};'' suspends computation and block the co-routine until the computation is resumed by another co-routine calling ``resume'' method of the present coroutine,
\item ``co_yield'' yields a value, and 
\item ``co_return'' returns a value. 
\begin{end}

A difficulty in using coroutines is the fact that the coroutine may live longer than the scope it has been called from. It is hence \emph{not} advisable to pass by reference. One can either pass by value or pass, e.g., ``std::unique_ptr'':

\raggedbottom
\begin{codebox}[]{}{}
\footnotesize An example of the use of coroutines, which currently does not compile in GCC 12.2.
\tcblower
\cppfile{code_examples/cpp23/coroutines2.cpp}
\end{codebox}

\raggedbottom
\begin{codebox}[]{\href{https://godbolt.org/z/1nYdMPh3z}{\ExternalLink}}
\footnotesize An example of the use of coroutines.
\tcblower
\cppfile{code_examples/cpp23/coroutines3.cpp}
\end{codebox}

\raggedbottom
\begin{codebox}[]{\href{https://godbolt.org/z/}{\ExternalLink}}
\footnotesize An example of the use of coroutines.
\tcblower
\cppfile{code_examples/cpp23/coroutines4.cpp}
\end{codebox}

\section{Synchronisation Primitives}

\subsection{Barrier}

\raggedbottom
\begin{codebox}[]{\href{https://godbolt.org/z/}{\ExternalLink}}
\footnotesize An example of the use of a barrier.
\tcblower
\cppfile{code_examples/cpp23/barrier1.cpp}
\end{codebox}

\raggedbottom
\begin{codebox}[]{\href{https://godbolt.org/z/}{\ExternalLink}}
\footnotesize An example of the use of a barrier.
\tcblower
\cppfile{code_examples/cpp23/barrier2.cpp}
\end{codebox}

\subsection{Atomic Variables}

\raggedbottom
\begin{codebox}[]{\href{https://godbolt.org/z/}{\ExternalLink}}
\footnotesize An example of the use of atomic variables.
\tcblower
\cppfile{code_examples/cpp23/atomic1.cpp}
\end{codebox}

\raggedbottom
\begin{codebox}[]{\href{https://godbolt.org/z/}{\ExternalLink}}
\footnotesize An example of the use of atomic variables.
\tcblower
\cppfile{code_examples/cpp23/atomic2.cpp}
\end{codebox}

\raggedbottom
\begin{codebox}[]{\href{https://godbolt.org/z/}{\ExternalLink}}
\footnotesize An elaborate example of the use of atomic variables.
\tcblower
\cppfile{code_examples/cpp23/atomic3.h}
\end{codebox}

\subsection{Mutexes}

\raggedbottom
\begin{codebox}[]{\href{https://godbolt.org/z/}{\ExternalLink}}
\footnotesize An example of the use of a barrier.
\tcblower
\cppfile{code_examples/cpp23/mutex1.cpp}
\end{codebox}

\raggedbottom
\begin{codebox}[]{\href{https://godbolt.org/z/}{\ExternalLink}}
\footnotesize An example of the use of a barrier.
\tcblower
\cppfile{code_examples/cpp23/mutex2.cpp}
\end{codebox}

\section{Algorithms in the Standard Template Library}

\subsection{For Each}

\raggedbottom
\begin{codebox}[]{\href{https://godbolt.org/z/}{\ExternalLink}}
\footnotesize An example of the use of for each.
\tcblower
\cppfile{code_examples/cpp23/for_each_code_cpp20.h}
\end{codebox}

\raggedbottom
\begin{codebox}[]{\href{https://godbolt.org/z/}{\ExternalLink}}
\footnotesize An example of the use of for each.
\tcblower
\cppfile{code_examples/cpp23/for_each_code_parallel.h}
\end{codebox}

\subsection{Reduce}

\raggedbottom
\begin{codebox}[]{\href{https://godbolt.org/z/}{\ExternalLink}}
\footnotesize An example of the use of reduce.
\tcblower
\cppfile{code_examples/cpp23/reduce_code.h}
\end{codebox}

\subsection{Merge}

\raggedbottom
\begin{codebox}[]{\href{https://godbolt.org/z/}{\ExternalLink}}
\footnotesize An example of the use of a merge.
\tcblower
\cppfile{code_examples/cpp23/merge_par_code.h}
\end{codebox}
